
\documentclass[12pt]{article}
\title{CH107 Assignment 2}
\author{Harsh Poonia}
\date{4 February 2022}
\usepackage[margin=1in]{geometry}
\usepackage{amsmath}
\begin{document}
\maketitle
\begin{abstract}
    This is my recap of Week 2 of course CH107 which covered Rigid Rotor and Analysis of the Hydrogen Atom.
\end{abstract}
\section*{Spherical Polar Coordinates}
Spherical polar coordinates can be used to locate any point in space, just like Cartesian coordinates.
The parameters used are $r$, $\theta$ and $\phi$. \\
$r$ is the euclidean distance from the point to the origin. In terms of cartesian coordinates, $r=\sqrt{x^2+y^2+z^2}$. 
$\theta$ is the angle between the position vector of the point and the $z$-axis. $\cos \theta = \frac{z}{r}$. $\theta$ lies from $0$ to $\pi$.
$\phi$ is the angle between projection of position vector on $xy$-plane and the $x$-axis. $\phi$ lies between $0$ to $2\pi$.
\[z=r \cos \theta, y=r \sin \theta \sin \phi, x= r \sin \theta \cos \phi\]
\[\text{Volume Element } dV = r^2 \sin \theta d\theta d\phi dr\]

\section{Rigid Rotor}
The basic premise of a rigid rotor is this - 2 massive bodies of masses $m1$ and $m2$ are freely revolving about their centre of mass and are constrained such that the distance between them is fixed. This problem is equal to the following simpler model - a single massive particle of mass $\mu $ is revolving in a circle of radius $r_0$ where $\mu$ is the \emph{reduced mass} of the system and $r$ is the \emph{radius of gyration} of the system about its centre of mass. 

Spherical coordinates are more convenient to use while solving this problem.
\subsection*{The Hamiltonian}
\subsubsection*{The Laplacian}
In polar coordinates, the laplacian takes the form
\[\nabla^2 = \frac{1}{r^2} \frac{\partial}{\partial r} \left(r^2 \frac{\partial}{\partial r}\right) + \frac{1}{r^2 \sin\theta} \frac{\partial}{\partial \theta } \left(\sin \theta \frac{\partial}{\partial \theta }\right) +  \frac{1}{r^2 \sin^2{\theta}} \frac{\partial^2}{\partial \phi^2 } \]
\subsubsection*{Kinetic and Potential energies}
\[T = -\frac{\hbar^2}{2\mu}\nabla^2\]
Potential energy for a rigid rotor is zero since distance between the bodies is invariant. \\ 
Thus, $\hat{T}=\hat{H}$. Also since $r$ is constant, $T$ is invariant w.r.t change in $r$. Derivatives w.r.t $r$ in the Laplacian are thus $0$.

We know, \[\hat{L}^2=-\hbar^2 \left[ \frac{1}{ \sin\theta} \frac{\partial}{\partial \theta } \left(\sin \theta \frac{\partial}{\partial \theta }\right) +  \frac{1}{\sin^2{\theta}} \frac{\partial^2}{\partial \phi^2 } \right]\]
\[T=\frac{L^2}{2I}, I=\mu r_0^2, \Rightarrow \boxed{\hat{H}= \frac{\hat{L}^2}{2\mu r_0^2 }}\]
\subsection*{The Wavefunction}
The wavefunction that represents the rigid rotor system is a \textbf{spherical harmonic} and thus variable separable in $\theta$ and $\phi$. 
\[Y_J^m=\Theta_{J, |m|}(\theta) \cdot \Phi_m(\phi)\]
Divide both sides of the \emph{eigenvlaue} equation $\hat{H}Y=EY$ by $\Theta \cdot \Phi$, multiply both sides by $\sin^2{\theta}$, to obtain 
\[-\frac{1}{\Phi}\frac{\partial^2 \Phi}{\partial \phi^2} = \beta^2 \sin^2\theta + \frac{\sin \theta}{\Theta} \frac{\partial}{\partial \theta} \left( \sin\theta \frac{\partial \Theta }{\partial \theta}\right) = m^2\]
where $\beta = \displaystyle \frac{2\mu r_0^2}{\hbar^2}$ and $m $ is a constant.

Solving for $\Phi$ and applying \textbf{boundary condition} that $\Phi(\phi + 2n\pi) = \Phi(\phi)$, we obtain
\[\Phi=A \exp{(\pm i m \phi)} \text{ where } m=0, \pm 1, \pm 2, \pm 3 \ldots\]
Change in $\phi$ corresponds to motion in the $xy$ plane, and we see that angular momentum along $z$ is thus \[\hat{L_z}=-i\hbar \frac{\partial}{\partial\phi}\]
For our rigid rotor, angular momentum along $z$ is 
\[\hat{L_z}\Phi=-i\hbar \frac{\partial}{\partial\phi} \left( Ae^{im\phi} \right) = \hbar m \Phi\]
This phenomenon is called \textbf{space quantisation}, ie. angular momentum along $z$ axis can only take certain specific values.

Solutions to $\Theta$ equation are a set of power series called \textbf{Associated Legendre Polynomials}, and require that $\beta= J(J+1)$ where $J \geq |m|$. Thus $(J, m)$ represent a specific wavefunction of the system with energy $\displaystyle E=\frac{\hbar^2 \beta}{2\mu r_0^2}= \frac{J(J+1)\hbar^2}{2I}$, clearly \textbf{quantised.} These degenerate enrgy states for the same $J$, but different $m$ differ in the \emph{orientation} of the angular momentum of the system.

\section{Hydrogen Atom}
\subsection*{Energy}
The energy of the hydrogen atom can be split into 3 parts. 
\[\hat{H}=\hat{T}_N+\hat{T}_e+\hat{V}_{N,e} = -\frac{\hbar^2}{2} \left( \frac{\nabla_N^2}{m_N} + \frac{\nabla_e^2}{m_e}\right) - \frac{1}{4\pi \epsilon_0} \frac{Ze^2}{r_{eN}}\]
The wavefunction here is a function of $6$ parameters-the coordinates of nucleus and electron.

Solving this system is much easier in the frame of the nucleus and using polar coordinates in that frame for the motion of electron.
Separating the wavefunction $\Psi= \phi_N \phi_e$ into nucleus and electron parts, and Hamiltonian into $\hat{H}_N$ and $\hat{H}_e$, we obtain that \textbf{nuclues behaves like a free particle in 3D space.} 
\subsection*{Wavefunction for Electron}
For $\phi_e$, we write the Laplacian in spherical coordinates, and separate $\phi_e$ w.r.t 3 variables $r, \theta$ and $\phi$.\[\Psi_e = R(r) \Theta(\theta) \Phi(\phi)\]
Using techniques for solving variable separable PDEs from rigid rotor, 
\[\hat{L}^2 Y(\theta, \phi) = (\hbar^2 \beta) Y(\theta, \phi) \text{ where } \beta = l(l+1)\]
\[\Phi=A \exp(im\phi) \Rightarrow \hat{L}_z \Phi = \hbar m \Phi\text{ and } \Theta = \text{ associated Legendre Polynomials}\]
\subsubsection*{Angular momentum}
The angular momentum in $z$ direction is again \textbf{space quantised} and total angular momentum is quantised with $|L|=\hbar \sqrt{l(l+1)}$. We have $l \geq |m|$. 

We cannot measure $\Vec{L}$ without disturbing its direction. 
\begin{quote}
    \textbf{It is important to emphasize that the three wavefunctions $R, \Phi$ and $\Theta$ aren't completely independent of each other. Their dependences arise out of boundary conditions. This is the reason why the Hamiltonian is not separable with respect to  $(r, \theta, \phi)$ coordinates.} 
\end{quote}
$R(r)$ is dependent on quantum numbers $n$ and $l$, $\Theta (\theta)$ is dependent on $l$ and $m$, while $\Phi(\phi)$ is dependent on $m$.
\end{document}
